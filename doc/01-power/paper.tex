\documentclass[11pt, letterpaper]{article}
	\title{Energy-Efficient Real-Time Systems}
	\author{Daniel Lo \{dl575\}}
	
	% Packages for math formatting
	\usepackage{amsfonts}
	\usepackage{amsmath}
	\usepackage{amsthm}

	% for letter paper
	\usepackage[letterpaper]{geometry}
	% 1 inch margins
	\usepackage{fullpage}    
	% allow enumerate numberings to be specified
	\usepackage{enumerate}
	% Insert images
	\usepackage{graphicx}
	% For syntax highlighting
	\usepackage{listings}
  % Graphviz
  \usepackage{graphviz}
  % Algorithms
  \usepackage{algorithm}
  \usepackage{algorithmic}
		
	% Header
	\usepackage{fancyhdr}
	\pagestyle{fancy}
	\headheight 30pt
	\rhead{}
	\lhead{Energy-Efficient Real-Time Systems \\
  Daniel Lo \{dl575@cornell.edu\}}
  \headsep 0.1in
	
	% Define a problem "theorem" heading
	\newtheorem{problem}{Problem}
	% Proposition theorem
	\newtheorem{proposition}{Proposition}
	% Scientific form
	\providecommand{\e}[1]{\ensuremath{\cdot 10^{#1}}}
	% Matrix - use mathbf font
	\providecommand{\m}[1]{\mathbf{#1}}
	% Degree symbol
	\providecommand{\degrees}{^{\circ}}
	% Insert figure
	\providecommand{\fig}[1]{
		\noindent
		\begin{center}
			\includegraphics[height=3.5in]{#1}
		\end{center}
	}
  \providecommand{\figcol}[1]{
    \noindent
    \begin{center}
      \includegraphics[width=\columnwidth]{#1}
    \end{center}
  }
	\providecommand{\dualfig}[2]{
		\noindent
		\begin{center}
		\includegraphics[width=3.2in]{#1} \hspace{-0.3in}
		\includegraphics[width=3.2in]{#2}
		\end{center}
	}
	\providecommand{\tripfig}[3]{
		\noindent
		\begin{center}
		\hspace{-0.3in}
		\includegraphics[width=2.3in]{#1} \hspace{-0.3in}
		\includegraphics[width=2.3in]{#2} \hspace{-0.3in}
		\includegraphics[width=2.3in]{#3} \hspace{-0.3in}
		\end{center}
	}
	% begin/end align*
	\providecommand{\eq}[1]{
		\begin{align*}
		#1
		\end{align*}
	}
	% Kernel/Image
	\providecommand{\im}[1]{
		\text{Im}(#1)
	}
	\renewcommand{\ker}[1]{
		\text{Ker}(#1)
	}
	% Trace
	\providecommand{\tr}[1]{
		\text{Tr}\left(#1\right)
	}
	% Such that
	\providecommand{\st}[0]{
		%\text{ s.t. }
		\ni
	}
	% Real numbers
	\providecommand{\reals}[0]{
		\mathbb{R}
	}
	% Use overline which is longer instead of bar
	\renewcommand{\bar}[1]{
		\overline{#1}
	}
	% Create divided matrix representing state-space system
	\providecommand{\statespace}[4]{
		\begin{bmatrix}
		\begin{array}{c|c}
			#1 & #2 \\
			\hline
			#3 & #4
		\end{array}
		\end{bmatrix}
	}
	% Include matlab code
	\providecommand{\matlab}[1]{
	  \lstinputlisting[language=matlab,
		showstringspaces=false,
		basicstyle=\footnotesize]
		{#1}
	}
  % Generate graph using dot/graphviz
  \providecommand{\dotgraph}[2]{
    \vspace{-0.6in}
    \begin{center}
    \digraph[scale=0.75]{#1}{#2}
    \end{center}
    \vspace{-0.6in}
  }
		
	
	% Insert a blank line between paragraphs
	\setlength{\parskip}{\baselineskip}
\begin{document}

%\maketitle

\section{Introduction}

This document discusses plans and ideas dealing with the idea of using
execution time prediction to inform the use of DVFS in soft real-time (e.g.,
interactive) systems.

\section{Potential Gains}

This section attempts to roughly model the power and energy savings that could
be gained using execution time prediction to inform DVFS. The goal is to see
what the best case improvements are.

\subsection{Power Modeling}

We need some way to model what the power/energy usage for executing a job at a
specified DVFS point is. Consider the following simple model of core power.

\eq{
  P_{static} =& I_{leak}V \\
  P_{dynamic} =& \alpha C V^2 f \\
  P =& P_{static} + P_{dynamic} \\
  =& I_{leak}V + \alpha C V^2 f 
}

Let's begin by only considering the dynamic power. The dynamic component of the energy used by a job that takes time $t$ is then
\eq{
  E_{dynamic} = \alpha C V^2 f t
}
If we assume that the job time is exactly proportional to frequency ($t = \beta/f$) then,
\eq{
  E_{dynamic} = \alpha \beta C V^2
}

\subsubsection{DVFS Data}

For the OMAP 3530 which runs a Cortex-A8
\footnote{http://www.ti.com/lit/ds/symlink/omap3530.pdf}, the following table
lists the recommended operating points.

\begin{tabular}{|l|r|r|}
\hline
Operating Point & Voltage [V] & Frequency [MHz] \\ \hline\hline
OPP6 & 1.35 & 720 \\ \hline   
OPP5 & 1.35 & 600 \\ \hline   
OPP4 & 1.27 & 550 \\ \hline   
OPP3 & 1.20 & 500 \\ \hline   
OPP2 & 1.06 & 250 \\ \hline   
OPP1 & 0.985 & 125 \\ \hline   
\end{tabular}

OPP6 is supported on high-speed versions of the OMAP 3530 and we will assume
this version for our analysis. The following is a graph of these DVFS
operating points.

% For the Intel Pentium M
% \footnote{http://download.intel.com/design/network/papers/30117401.pdf}, the
% DVFS operating points are listed in the following table.
% 
% \begin{tabular}{|l|r|r|}
% \hline
% P-State & Voltage [V] & Frequency [MHz] \\ \hline
% P0 & 1.484 & 1600 \\\hline
% P1 & 1.420 & 1400 \\ \hline
% P2 & 1.276 & 1200 \\ \hline
% P3 & 1.164 & 1000 \\ \hline
% P4 & 1.036 & 800 \\ \hline
% P5 & 0.956 & 600 \\ \hline
% \end{tabular}

\fig{dvfs.pdf}

We normalize energy usage and execution times to the values at OPP6. These
are shown in the following table and graph. 

% OMAP 3530
\begin{tabular}{|l|r|r|}
\hline
Operating Point & Normalized Energy & Normalized Execution Time \\ \hline\hline
OPP6 & 1.00 & 1.00 \\ \hline
OPP4 & 0.88 & 1.31 \\ \hline
OPP3 & 0.79 & 1.44 \\ \hline
OPP2 & 0.62 & 2.88 \\ \hline
OPP1 & 0.53 & 5.76 \\ \hline
\end{tabular}

% Intel Pentium M
% \begin{tabular}{|l|r|r|}
% \hline
% Operating Point & Normalized Energy & Normalized Execution Time \\ \hline\hline
% P0 & 1.00 & 1.00 \\ \hline
% P1 & 0.92 & 1.14 \\ \hline
% P2 & 0.74 & 1.33 \\ \hline
% P3 & 0.62 & 1.60 \\ \hline
% P4 & 0.49 & 2.00 \\ \hline
% P5 & 0.41 & 2.67 \\ \hline
% \end{tabular}

\fig{dvfs_energy.pdf}

\subsubsection{Execution Time Data}

The following is the execution time for performing video conversion on each
frame of ironman3.mp4 using ffmpeg. This was run on a 2013 Macbook Air running a 1.7 GHz Intel Core i7. The minimum time is 2.9 ms. The average
time is 4.9 ms. The maximum time is 39.7 ms. There are 3659 frames.
\figcol{frame_time.pdf}

We will assume that this data corresponds to operating at OPP6 for the OMAP
3530. In addition, we will assume that the execution time requirement is the
worst-case achieved time of 40 ms.

\subsection{DVFS Schemes}

\textbf{Naive Schemes: }
A naive scheme to meet the execution time would be to set the DVFS point at the
fastest operating point in order to account for the worst-case frame time. Let
$E_i = 1$ be the energy for processing frame $i$ at P0, then the total energy
usage for this worst-case scheme is $E = 3659$ and no frames violate the timing
requirement. We could use similarly naive schemes that use a constant DVFS
point. The energy usage and percentage of frames which violate timing are
listed in the following table.

\begin{tabular}{|l|c|c|}
\hline
Operating Point & Energy & Frames Violated \\ \hline\hline
OPP6 & 3659 (100\%) & 0 (0\%) \\ \hline
OPP4 & 3238 (88\%) & 1 (0.03\%) \\ \hline
OPP3 & 2891 (79\%) & 1 (0.03\%) \\ \hline
OPP2 & 2255 (61\%) & 3 (0.08\%) \\ \hline
OPP1 & 1947 (53\%) & 40 (1.09\%) \\ \hline
\end{tabular}

\textbf{Optimal (Oracle) Scheme: } 
If we choose the lowest voltage setting for each frame such that the 40ms
timing requirement is still met, then the total energy is 1952 which is 53\% of
the energy usage of running all frames at OPP6. This gives the energy of
running at OPP1 but with no frames violated as in running at OPP6.

Note that with the DVFS settings we explore here, for the average frame which
ran in 4.9ms at OPP6, its execution time at OPP1 is 28.2ms which is still
within our timing requirement. With lower DVFS points (or tighter timing
requirements), greater energy savings are possible.

\textbf{Adaptive Scheme: }
A simple adaptive scheme is to adapt the DVFS setting based on previous frame times.
Using only information from the previous frame, the total energy of 1952 (53\%)
but has 30 violated frames (0.82\%). This is not surprising looking at the
execution time graph as the execution times do not show phase behavior.

% \section{Prediction Design Space}
% 
% \section{Previous Work on DVFS}

\end{document}
