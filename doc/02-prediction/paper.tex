\documentclass[11pt, letterpaper]{article}
	\title{Designs and Approaches for Execution Time Prediction}
	\author{Daniel Lo \{dl575\}}
	
	% Packages for math formatting
	\usepackage{amsfonts}
	\usepackage{amsmath}
	\usepackage{amsthm}

	% for letter paper
	\usepackage[letterpaper]{geometry}
	% 1 inch margins
	\usepackage{fullpage}    
	% allow enumerate numberings to be specified
	\usepackage{enumerate}
	% Insert images
	\usepackage{graphicx}
	% For syntax highlighting
	\usepackage{listings}
  % Graphviz
  \usepackage{graphviz}
  % Algorithms
  \usepackage{algorithm}
  \usepackage{algorithmic}
		
	% Header
	\usepackage{fancyhdr}
	\pagestyle{fancy}
	\headheight 30pt
	\rhead{}
	\lhead{Designs and Approaches for Execution Time Prediction \\
  Daniel Lo \{dl575@cornell.edu\}}
  \headsep 0.1in
	
	% Scientific form
	\providecommand{\e}[1]{\ensuremath{\cdot 10^{#1}}}
	% Matrix - use mathbf font
	\providecommand{\m}[1]{\mathbf{#1}}
	% Degree symbol
	\providecommand{\degrees}{^{\circ}}
	% Insert figure
	\providecommand{\fig}[1]{
		\noindent
		\begin{center}
			\includegraphics[height=3.5in]{#1}
		\end{center}
	}
  \providecommand{\figcol}[1]{
    \noindent
    \begin{center}
      \includegraphics[width=\columnwidth]{#1}
    \end{center}
  }
	\providecommand{\dualfig}[2]{
		\noindent
		\begin{center}
		\includegraphics[width=3.2in]{#1} \hspace{-0.3in}
		\includegraphics[width=3.2in]{#2}
		\end{center}
	}
	\providecommand{\tripfig}[3]{
		\noindent
		\begin{center}
		\hspace{-0.3in}
		\includegraphics[width=2.3in]{#1} \hspace{-0.3in}
		\includegraphics[width=2.3in]{#2} \hspace{-0.3in}
		\includegraphics[width=2.3in]{#3} \hspace{-0.3in}
		\end{center}
	}
	% begin/end align*
	\providecommand{\eq}[1]{
		\begin{align*}
		#1
		\end{align*}
	}
	% Use overline which is longer instead of bar
	\renewcommand{\bar}[1]{
		\overline{#1}
	}
  % Generate graph using dot/graphviz
  % First argument is graph name
  % Second argument is the dot notation of a graph, e.g.:
  %   rankdir=LR; a->b
  \providecommand{\dotgraph}[2]{
    \vspace{-0.6in}
    \begin{center}
    \digraph[scale=0.75]{#1}{#2}
    \end{center}
    \vspace{-0.6in}
  }
		
	% Insert a blank line between paragraphs
	\setlength{\parskip}{\baselineskip}
\begin{document}

%\maketitle

\section{Introduction}

This document discusses approaches and trade-offs for performing execution time
prediction (in the context of using it to inform DVFS).

\textbf{Problem: } Given a set of data values, we want to predict the execution
time of a job. The problem includes determining which data values are necessary
as well as the algorithm for performing the prediction. We will assume that we
have some number of jobs that are used for training. Training can include using
certain jobs or operating points (i.e., DVFS setting) to aid the training. In
addition, once training is complete, information from running jobs can be used
to further improve the prediction model.

\section{Machine Learning Background}
\label{sec:ml}

This section attempts to summarize some of the problems and
algorithms/approaches addressed by machine learning which may be of interest
for predicting execution times. The information is a combination of information
from Wikipedia and personal knowledge.

\subsection{Problems}
\label{sec:ml.problems}

There are two machine learning problems which are of interest to us in
predicting execution times: \emph{classification} and \emph{regression}.

\noindent\textbf{Regression: } Regression is the problem of identifying the relationship between independent
and dependent variables. In our case, the execution time prediction problem can
be framed as a regression problem to determine the relationship between
execution time and input data values. The resulting solution is an equation
relating the two which can be used to predict new data points.

\noindent\textbf{Classification: } is the problem of determining which category an item belongs to.
The execution time prediction problem can be framed as a classification problem
where the categories are time bins that the execution time can fall in. For
example, two simple categories are whether the execution time is slower or
faster than the required response-time threshold. Alternatively, categories
could be which minimal-power DVFS setting is required to meet the response-time
threshold.

\subsection{Algorithms}
\label{sec:ml.algs}

\noindent\textbf{Decision tree: } A decision tree is a tree used for classification
where each leaf node indicates a category and each non-leaf node indicates a
decision. A new data point can travel the decisions in the tree to reach a
category prediction. There exist various algorithms for constructing and
optimizing the tree. Decision trees require a discrete decision to be made at
each node (e.g., yes/no, greater than/less than threshold). As a result, it
seems to be a poor fit for execution time prediction as it would require
translating the input values into decisions.

\noindent\textbf{k-Nearest Neighbors (k-NN): } k-NN classifies a data point based on its
k-nearest neighbors. Regression can also be performed by taking an average of
the dependent variable value of the k-nearest neighbors. This is a relatively
simple algorithm. The main challenge is in identifying the nearest neighbors. A
naive solution requires checking all points, though better algorithms exist by
organizing existing points in a tree. k-NN can suffer from local structure of
the data and can also be dominated by categories which have a larger number of
data points.

\noindent\textbf{Linear Regression: } Linear regression is a simple version of
regression where a linear relationship is assumed between the dependent
variables and the independent variables. A linear model can be fit by using the
method of least-squares. This is effectively a matrix multiplication which
produces a linear equation that can be used to predict the execution time of
new jobs.

\noindent\textbf{Polynomial and Nonlinear Regression: } There also exist algorithms for
fitting polynomial and nonlinear curves to data. Polynomial regression can be
solved similar to linear regression through a set of matrix multiplications.
Nonlinear regression models are more difficult to solve for, typically
requiring some numerical approximation method or by approximating certain
things as linear and then refining.

\noindent\textbf{Artificial Neural Networks: } ANNs are neuron-inspired structures that
have been shown to be effective for pattern recognition patterns. The basic
element is a neuron which consists of some inputs and an output. In the
simplest model, the output ``fires'' when the sum of the inputs exceed some
threshold. A network of these is used to produce an output (typically a
classification). Training involves determining the weights of inputs and
thresholds to use for the neurons in these networks. ANNs have been shown to be
successful at pattern recognition problems such as handwriting recognition that
have been difficult for traditional algorithms. However, they can be
difficult/expensive to train and the computation required can be high due to
the need to simulate many neurons.

\noindent\textbf{Support Vector Machines: }  SVMs in their simplest form are binary
linear classifiers. Given a training set of labeled data, they attempt to
construct a plane to divide the space into two categories. This can be applied
to execution time prediction to predict whether jobs are slower or faster than
a threshold. More complex forms of SVMs can classify into more than 2
categories, perform non-linear classification, and perform regression. Training can
be slow/expensive as it relies on solving a quadratic programming problem.

The following table summarizes whether these approaches solve regression or
classification problems. Items in parentheses indicate that extensions exist to
solve these class of problems.

\begin{tabular}{|l|l|}
\hline
Algorithm & Problem Type \\ \hline\hline
Decision Tree & Classification \\ \hline
k-NN & Classification (Regression) \\ \hline
Linear Regression & Regression \\ \hline
Polynomial and Nonlinear Regression & Regression \\ \hline
Artificial Neural Networks & Classification \\ \hline
Support Vector Machines & Classification (Regression) \\ \hline
\end{tabular}

\section{Design Choices}

The following attempts to discuss some of the design choices that can be made
in creating a method for predicting execution time.

\noindent\textbf{Static Analysis: } Static analysis of source code can be used to find
data values that are useful for prediction. However, this does require
pre-processing and access to source code. Static analysis of the binary may
provide useful information without the need for source code.

\noindent\textbf{Offline vs. Online Training: } Whatever prediction mechanism is used
will require some amount of training data in order to calibrate and learn the
characteristics of an application. This training can be done either online,
offline, or some combination of the two. Offline training can use more
sophisticated techniques for training as its overheads do not contribute to the
overheads of the application. However, it requires more programmer effort.
Online training does not require extra programmer effort. It also is likely
more easily applied to new applications as it does not require a separate
training phase. In addition, it allows continuous updating of the prediction model.
A hybrid approach would allow an offline-trained model to be updated as the
application runs.

\noindent\textbf{Feature Extraction: } A large challenge is determining which features
(i.e., inputs, data values, metrics, etc.) should be used as inputs for
prediction. This can be done entirely by the programmer or by automated
processes or by some combination (e.g., programmer hints). Using the programmer
to indicate features requires more programmer effort but likely has better
results as they can provide domain-specific knowledge. An automated process can
be more generally applied, possibly even to cases where source code is not
available. However, it may require significant analysis time and prediction
accuracy may be lower than a programmer-based approach.

\noindent\textbf{Model Complexity: } In order to predict the execution time jobs, we are
essentially building a model to relate the features to the execution time. More
complex models will hopefully give better accuracy. However, they will likely
require more computation and thus introduce performance and energy overheads.
Less complex models introduce less overheads but may be inaccurate. Since our
goal is to save energy, this is an important point since we need to be accurate
enough to be able to find energy savings when they exist, but the overheads of
prediction need to not outweigh the energy savings.

\section{Limits of Prediction}
\label{sec:limit}

In this section, we study how much variation there is for frames across
multiple runs. The following figure shows 10 runs of ffmpeg with the same
input. This was run on a 2013 Macbook Air. All (obvious) user applications were
terminated and wi-fi was disabled in an attempt to reduce competing processes.

\figcol{figs/frames_overlay.png}

At this granularity, it seems that there is some correlation in execution time
for the average-case behavior. However, it is interesting to note that outliers
do not seem to be consistent across runs.

The following figure shows the difference between the execution time of each
frame on an individual run and the median execution time across runs. The
median was used instead of the average in order to reduce the effects of
outliers. One arbitrary run (Run \#5) is highlighted in red. The second graph
zooms in on the y-axis to be between -4 ms and +4 ms.

\figcol{figs/frames_diff.png}

\figcol{figs/frames_diff_reduced.png}

There looks to be quite a bit of variation across runs including some large
outliers. Most of the variation looks to be within 1ms of the median across
runs. This inherently puts a limit to how accurate our prediction mechanism can
be.

\section{Related Work}

We discuss some previous work that performs execution time prediction based on
input values.

Zhu and Reddi \footnote{Yuhao Zhu and Vijay Janapa Reddi. High-Performance and
Energy-Efficient Mobile Web Browsing on Big/Little Systems. HPCA 2013.} use a
restricted cubic spline-based model to model webpage load time and
energy consumption based on 376 input variables. These input variables are
selected from extensive profiling and micro-benchmarking. They use 2,500 sample
to train the model. To prevent overfitting, they eliminate features which
correlate poorly ($\rho^2 < 1$).  They also prune features which are highly
correlated to each other. They also evaluate a linear regression model but the
cubic spline-based model shows better prediction accuracy.

ATLAS \footnote{Michael Roitzsch et al. ATLAS: Look-Ahed Scheduling Using
Workload Metrics. RTAS 2013.} uses a linear regression model to predict
execution times based on input
values. This model is continuously updated as new jobs are executed. In
addition, an ageing mechanism decreases the impact of older jobs on the model.
Input metrics are identified by the programmer and larger metric values should
correlate with higher execution times. They evaluate two sets of metrics. The
first is a reduced set including the number of pixels in a frame, the number of
bytes in the compressed frame, and the frame type  (I, P, or B). They also
evaluate using a more extensive set of metrics based on programmer knowledge
which are embedded into the video data through a pre-processing step (based on
their previous work \footnote{Roitzsch and Pohlack. Principles for the
prediction of video decoding times applied to MPEG-1/2 and MPEG-4 part 2 video.
RTSS 2006.} \footnote{Roitzsch. Slice-balancing h.264 video encoding for
improved scalability of multicore decoding. EMSOFT 2007.}). In theory these
metrics could also be calculated at run-time by rewriting the video decoder to
first calculate these metrics.


Mantis \footnote{Yongin Kwon et al. Mantis: Automatic Performance Prediction
for Smartphone Applications. ATC 2013.} predicts execution time based
ultimately on program inputs, but uses
intermediate values calculated in the code as the direct features for
prediction. They use a multivariate polynomial regression model. This model is
found by first using sparse linear regression to eliminate a number of
features. They then perform sparse multivariate polynomial regression on the
remaining features. Specifically, their sparse regression is done using the
LASSO (Least Absolute Shrinkage and Selection Operator) method. The features
they use are loop counts, method invocation counts, branch counts, exception
counts, and the first k values assigned to each variable. The sparse regression
analysis eliminates most of these features. The remaining features are used to
find a minimal program slice. Thus, at runtime, the minimal program slice can
be used to calculate the relevant feature values, which can then be fed into
the regression model to predict execution time. There is an iterative process
in order to minimize the program slice execution time and maximize the
prediction accuracy.

The following table attempts to summarize the prediction mechanisms used in these studies.

\noindent
\begin{small}
\begin{tabular}{|l||l|l|}
\hline
Work & Prediction Model & Features \\\hline\hline

Zhu \& Reddi & Restricted Cubic Spline-based Regression & Values selected through extensive profiling \\ \hline
ATLAS & Linear Regression & Programmer specified values \\ \hline
Mantis & Sparse Polynomial Regression & Automatically profiled values \\ \hline
\end{tabular}
\end{small}

\section{Naive Approaches}

\textbf{Utilization-based Approach: } This approach increases DVFS when the CPU
is heavily utilized. This will likely perform poorly in terms of energy saving
since during video decode the system will see that there is computation to be
done and use the highest DVFS setting.

\noindent\textbf{Previous or Time-Averaged Approach: } This approach uses information
from the last n jobs to inform the setting for the current job. From our
experimental data, we have seen large variation in execution time from job to
job, indicating that using previous history is not enough to perform a good
prediction.

\section{Proposed Approach}

\subsection{Feature Extraction}

% \dotgraph{input_variation}{
%   rankdir=LR;
%   "input data"->"control flow"->"execution time"; 
%   "input data"->"uarch state"->"execution time";
%   "input data"->"variable time inst"->"execution time";
%   "interfering process"->"execution time"; 
%   "interfering process"->"uarch state";
%   "previous uarch state"->"execution time"
% }
% 
% We expect that most of the variation in execution time is due to different
% input data. Here, we define input data as any data that is used by the
% computation of the job that can have different initial values for different job
% instances. This includes state that may be saved from previous jobs (e.g.,
% frame number) but does not include intermediate calculated values which vary
% from job to job. The other possible sources of variation are competing
% processes and microarchitectural state. Other processes may compete for CPU
% time and other shared resources which can impact a job's execution time.
% We ignore this initially though there may be opportunities to account for this
% in the prediction or to reduce this interference through scheduling mechanisms.
% Microarchitectural state, such as cache state, can also cause jobs to have
% different execution times. We expect this effect to be minimal on average and
% will ignore it for now.

In order to be general and easily applicable to new applications, we avoid
having the programmer specify relevant features for execution time prediction.
Instead, we will attempt to automatically identify features.

Although it should be possible able to determine the execution time from input data, the
function mapping input data to execution time may be complex. For example, one
way to calculate the execution time from the inputs is to simply run the job
and measure the execution time. However, in order to be useful, we need a
faster and lower overhead method. We could simply use the input data and try to use various
regression models, but this is unlikely to be successful due to the large
and possibly complex ways in which the inputs relate to execution time.
However, we do have some information about how these values
affect program execution since we have the program source. From traditional
WCET estimation, the most important factor in execution time variation is
control flow decisions. For certain control flow, it should be relatively easy
to correlate them to execution time. For example, for simple for loops, we can
determine the number of loop iterations based on the control flow variables.
Similarly, for if statements, by knowing which branch is taken and from
execution time data, we should be able to correlate which direction is longer.
However, more complicated control, such as iterating over all bytes in a file,
may be difficult to calculate without actually performing the action.
Once these values are identified, we can use a program-slicing approach,
similar to Mantis, in order to generate a simple way to calculate these values.

% A major challenge seems to be determining which data is most relevant to
% execution time. Previous work has used domain-specific knowledge to identify
% possible candidates and still requires experimentation to determine which exact
% features are best. One reason this is challenging is because the amount of
% input data can be large. For video decoding, it is the entire frame's data. For
% web browsing, it is the entire web page data. In addition, transforming the raw
% data into sensible metrics can be challenging. For video decoding, the raw data
% is compressed and likely needs to be decompressed in order to be of any use.

\subsection{Prediction Model}
From the data in Section~\ref{sec:limit}, there is considerable noise in the
execution time across runs. This can cause challenges for classification-based
approaches for prediction. Classification approaches attempt to divide the
input space into distinct categories. However, noise could change which
category a data point will fall into. We saw these issues with our initial
SVM-based prediction experiments where thresholds around the average value
showed much lower prediction accuracy. From the algorithms presented in
Section~\ref{sec:ml.algs}, this eliminates decision tress and ANNs. k-NNs and
SVMs could work by using regression-based versions of them. Linear and
polynomial regression seem to be the most promising as the method of least
squares is a computationally simple way to solve for these models.

On the other hand, classification-based approaches may make sense as there are
typically discrete choices of DVFS settings. Our ultimate goal is to classify
jobs into the DVFS bin to use. If classification-based approaches are used, it
may make sense to use (or at least first use) k-NN in order to discover the
underlying categories.

\subsection{Possible Relaxations}

There are possibly some ways to relax the problem that would make the
prediction easier. One is that we could allow some portion of the job's code to
execute before we make our prediction and our DVFS decision. For example, the
execution time to decompress a video frame may correlate well with the
execution time of the later decoding steps. Similarly, it may be that certain
(early) control decisions quickly split the execution times into high and low
clusters. We could wait until this branch decision to make our DVFS decision.
Similarly, we could also attempt to use microarchitectural metrics (e.g., cache
miss rate, branch mispredict, etc.) related to executing a portion of the job
to aid in the prediction.

% Another point is that we could assume that many consecutive jobs are run. In
% this case, we can use the execution times of previous jobs as a baseline and
% focus on predicting variation from this. This helps deal with macro-level
% variations. For example, this could be used to handle the effect of competing
% processes. Also, this could be used to handle videos of different resolutions
% without having to predict the effect of resolution on frame decode time. If the
% decode time of all frames scales proportionally to the resolution, then we can
% use the first frame as a baseline and focus on the properties that affect
% variation among frames in a single video.

\end{document}
